\subsection{Working of LIGO}

\subsubsection{Interference}

The name “Interferometer” comes from the principle of working itself i.e. interference.  Interference simply means interfere. When two waves, whether it is a light wave or a water wave, interfere with each other they add up and give some result. That is known as interference pattern. When the crest(trough) of the first wave interfere with crest i.e. peak(or trough i.e. dip) of the other, they interfere constructively. But when the crest falls on the trough of the other wave, it is known as destructive interference. But these are not the only cases. The crest may not fall only on the crest or trough of the other. When watching it graphically, it may be touching somewhere at the middle of the other wave. The simple way of counting it is adding up the amplitude of both the waves. 
\[ A_{1} + A_{2} = A_{12} \]
The resultant amplitude decides the intensity of light that will be produced finally. Intensity of light is directly proportional to the square of the amplitude.
\[ A_{2} \propto I \]
When the resultant amplitude is zero, the intensity is also zero and hence no light is observed. But when the amplitude is other than zero, the intensity is also greater than zero and some light is detected or observed.
\subsubsection{Lasers}
How the laser light is produced and amplified is already well described in the construction part. So, how laser light travels in the whole interferometry will be discussed here. To operate LIGO at its fullest potential, we need an immensely powerful laser. In order to obtain such a laser beam, it must undergo multiple stages of amplification. An Nd: YAG laser after passing through Non-Planar Ring Oscillator (NPRO), Master-Oscillator Power Amplifier (MOPA), which consists of four thin rods of Nd: YVO4, and finally through High-Power Oscillator (HPO), a powerful laser of power 200W, with wavelength 1064nm is obtained. This light beam incidents on a power recycling mirror followed by a beam splitter. The partially reflecting beam splitter splits the light into the 4km long ultra-high vacuum chambers orthogonal to each other. Fabry-Perot cavities are included in these chambers to improve the sensitivity. The light circulates for around 300 times between the test mass mirrors in these Fabry-Perot cavities. This boosts the laser power to 100kW. 
The light in both cavities, after circulating, should recombine at the beam splitter leaving the arms. These light waves will be in coherence; hence they cancel out each other letting no light to reach the photo-detector. However, this is not the scenario if a gravitational wave passes nearby.

\subsubsection{Effect of gravitational waves on the interference pattern}

Generally, gravitational waves stretch the space in one direction and compress it in the normal direction. Hence, when the gravitational wave passes through the LIGO, the orthogonal arms will be subjected to a strain, i.e. their lengths are increased or decreased. If one cavity gets extended, the other cavity gets shortened. This causes a variation in the effective length travelled by the laser beam. The light from shorter arm reaches the beam splitter prior to the light from longer arm creating a small phase difference. Due to this resonance occurs and some light reaches the photo-detector. This gives out the signal indicating a gravitational wave. The light which did not reach the photo-detector returns to beam splitter followed by power recycling mirror to boost the laser.
But apart from gravitational waves, there are other disturbances also. Like earthquakes, trucks going through a road nearer to the interferometry can also cover the influence of GW on laser light which is as minute as 1/10000th the width of a proton. And hence to avoid this as much as they can, the whole system is evacuated to the trillionth part of the outside atmosphere. Other measures are also taken to dampen these vibrations which are described in this paper later.
 
\pagebreak


















































































\pagebreak

\documentclass[11pt, a4paper]{article}
\usepackage[a4paper, total={6in, 9.5in}]{geometry}

\title{\textbf{\Huge Gravitational Wave Physics and LIGO}}

\usepackage{amsmath, tensor, gensymb, graphicx, hyperref, authblk, amsfonts, amssymb, verbatim, wrapfig, subfig}
\usepackage[T1]{fontenc}
\usepackage[utf8]{inputenc}
\usepackage[english]{babel}
\usepackage[bottom]{footmisc}
\usepackage[sorting = none]{biblatex}

\hypersetup{
    colorlinks=true,
    linkcolor=blue,
    urlcolor=blue,
    citecolor=black}
\urlstyle{same}

\addbibresource{bibliography.bib}

\author[1]{Sagar JC}
\author[1]{Aditya Srivatsa}
\author[1]{Ajith U Shanbhag}
\author[2]{Kushal J}
\author[3]{Hardik Medhi}
\author[4]{Bhavya Goradia}
\author[5]{Kshitija Kshirsagar}
\author[6]{Tasmi Memon}
\author[7]{Samrudhi R Kanjarpane}
\author[8]{Ajay Atwal}
\author[9]{Divyanshi Agrawal}
\author[9]{Kirti Jadhav}
\author[9]{Ved Deshpande}
\author[10]{Dhyan Gandhi}

\affil[1]{St. Joseph's College, Bengaluru}
\affil[2]{Christ Junior College, Bengaluru}
\affil[3]{REVA University, Bengaluru}
\affil[4]{KJ Somaiya College of Engineering, Mumbai}
\affil[5]{Institute of Science, Nagpur}
\affil[6]{Maharaja Sayajirao University, Vadodara}
\affil[7]{Poornaprajna College, Udupi}
\affil[8]{University of Hyderabad, Hyderabad}
\affil[9]{St. Xavier's College, Mumbai}
\affil[10]{Manipal Institute of Technology, Udupi}

\renewcommand\Authands{ and }

\date{\today}

\begin{document}
\maketitle

\section*{Acknowledgement}
A special thanks to Naxxatra Science for giving us the opportunity to write a collaborative review paper. We would like to express our gratitude to Ms. Sitara Srinivasan, head of Naxxatra Science, Mr. Vikranth Pulamathi and Mr. Jishnu P Das for teaching us \LaTeX.


\pagebreak
\begin{center}
    \section*{Abstract}
\end{center}

\input{0 Abstract}

\providecommand{\keywords}[1]
{
  \small	
  \textbf{\textit{Keywords---}} #1
}

\keywords{General Relativity, Space time, Tensors, indices, Field equations, Wave equation, Polarization, Energy Flux, Doppler effect, Inverse square law, Inspiral mechanism, Black holes, Neutron stars, Pulsars, Revolving Binary system, LIGO, Interference, Coherency, Laser}
\pagebreak

\tableofcontents
\pagebreak

\section{Introduction}
\hspace{0.5cm} General relativity is an important part of physics which helps us understand our universe in a large scale like black holes, gravitational waves and our expanding universe. It describes gravity as a property of space and time rather than a force as mentioned by Newton. It tells us that the curvature of space-time is related to energy and momentum which is present inside matter. The wrapping of space-time as described by Einstein's field equations theorizes various phenomenons such as gravitational waves. It helps us to understand the region of space having black holes or neutron stars, or system of dynamical heavy masses which causes great changes in gravity.\\ We know that there are three dimensions of space which we can interact with. But Einstein include a fourth dimension namely time. The space-time consists of four dimensional space with three spatial dimensions and a time dimension. In a space, with no matter, the space-time is flat, i.e shortest distance between two points is a straight line, but in presence of matter, shape(curvature) of space-time will be altered, then shortest distance between two points will be a curved line like an hyperbola. It kind of produces a dent in the curvature of space-time. 

\begin{figure}[h]
    \centering
    \includegraphics[scale=0.2]{gw-physics-and-ligo/images.tex/illus_3dspace1.jpg}
    \caption{Curved space-time in presence of mass. Source:- \href{https://www.forbes.com/sites/startswithabang/2019/02/16/ask-ethan-how-can-we-measure-the-curvature-of-gravity/?sh=13c20ec1134f}{Forbes.com}}
\end{figure}

If there are two giant masses orbiting around each other, then it results into the formation of ripples in space-time which we call it as gravitational waves. The ripples move with the speed of light. As they move in the space, the farther they move away from the source or their origin, the more weak the get. Their strength decreases as they go away from the source. The gravitational waves which are detected are weak due to various reasons, but are good source of information.\\ In 1916, Einstein predicted that two bodies orbiting each other would not be in the same orbit all the time, instead the bodies lose energy and in doing so emit gravitational waves. According to his mathematics he showed that two massive accelerating objects such as neutron stars or black holes, when orbit around each other, it generates ripples and disrupt the space time such that waves are propagated in all the direction away from the source. This ripples carry the information about their origin and about gravity as well. The strongest gravitational waves are generated due to colliding black holes or neutron star. After half a century, the first indirect proof of gravitational waves was given in the year 1974 when the astronomers Jocelyn Bell and Antony Hewish discovered a pulsar which produces a gravitational wave. After this they were observing how the stars change its orbit as the time passes. They observed that the stars are getting closer to each other at the rate which was predicted by Einstein in his General Relativity which results into production of GW.\\ On September 14, 2015, LIGO sensed a signal which was due to the collision of two massive black holes which are 1.3 billion light years away from Earth. Later after analyzing the phenomena, it was observed that the wave was caused due to objects 29 and 36 times massive than the sun orbiting with 210 kilometer around each other before the collided. The gravitational waves which are generated near the source are very large but by the time they were detected here on earth, their strength was so small that the effect of GW on the detector was 1000 times smaller than the nucleus of an atom.

\pagebreak



\section{Linearized theory of Gravitational waves}
Linearized theory of Gravitational waves is a basic understanding of gravitational waves based on an assumption that any perturbation in space-time can be approximated to a linear factor whose degree is One. This simplifies the calculations a lot. And more over since the sources of gravitational waves are very far away, the effects they produce here on earth will be very small. So we can neglect the higher degree of perturbation and linearize it to the first degree. \\

Einstein's field equations are a set of ten Tensor equations which describe gravity as a curvature in space-time. And one among them is: 
\begin{equation}
    G_{\mu\nu}= \frac{8 \pi  G}{c^{4}}  T_{\mu\nu}
\end{equation}
This is a tensor equation which describes gravity in form of Einstein's tensor, $G_{\mu\nu}$ which is directly dependent on the geometry of space-time which is altered by the stress-energy tensor $T_{\mu\nu}$. Another field equation that relates the geometry or curvature of space-time to stress-energy tensor is 

\begin{equation}
    R_{\mu\nu}-\frac{1}{2}g_{\mu\nu}R=\frac{8\pi G}{c^{4}}T_{\mu\nu}
\end{equation}
where $R_{\mu\nu}$ is the Riemann tensor which describes the curvature of space-time, $R$ is the scalar curvature and $g_{\mu\nu}$ is the gravitational field tensor. Any change in matter distribution will be recorded in in $T_{\mu\nu}$. So if $T_{\mu\nu}$ changes then according to equation 2, gravitational field tensor $g_{\mu\nu}$ also has to change. If $h_{\mu\nu}$ is the perturbation induced in space-time then the new gravitational field tensor $\tilde{g}_{\mu\nu}$ is given by 

\begin{equation}
    \tilde{g}_{\mu\nu} = g_{\mu\nu} + h_{\mu\nu}
\end{equation}

\noindent To get the new gravitational field, field equation should be solved for $\tilde{g}_{\mu\nu}$ which gives 

\begin{equation}
    \tilde{h}_{\mu\nu} = h_{\mu\nu} - \frac{1}{2} \, \eta_{\mu\nu} \, h^{\alpha}_{\alpha}
\end{equation}
 where $\eta_{\mu\nu}$ is the gravity where space is flat i.e. $\eta_{\mu\nu} = g_{\mu\nu}$ and $h^{\alpha}_{\alpha}$ is summed for all spatial coordinates i.e. $\alpha$ takes values $(1,2,3) $ which corresponds to $(x,y,z)$. The admitted solutions for this variations in space time $\tilde{h}_{\mu\nu}$ has solution in the form of 
 
 \begin{equation}
     \tilde{h}_{\mu\nu} = A^{\mu\nu}\, e^{ik_{\alpha}x^{\alpha}}
 \end{equation}
 
 \noindent This is a 3D wave equation where $A^{\mu\nu}$ is the Amplitude tensor, $i = \sqrt{-1} $, $k_{\alpha} = (k_{x},k_{y},k_{z})$ is the wave vector and $x^{\alpha} = (x^{1},x^{2},x^{2}) = (x,y,z)$ is the position vector.
 \\
 
 Thus we can say that when ever a body causes disturbances in curvature of space-time, these travel through space in form of waves whose speed is equal to the speed of light.  
 
 \begin{figure}[h]
     \centering
     \includegraphics[scale=0.1.5]{images.tex/gw_representation.png}
     \caption{This is a computer simulated image that shows Gravitational wave as a 3D wave\\ Source:- \url{https://www.universetoday.com/127255/gravitational-waves-101/}}
 \end{figure}


\section{Properties of Gravitational waves}
Now we shall see the properties of these gravitational waves. like any other waves, even gravitational waves have frequency, wavelength, speed which is speed of light, etc. These properties depend on the source of gravitational waves. 
\subsection{Polarization of Gravitational waves}


Gravitational waves can also be. Since they are three dimensional waves their polarization can be restricted to two forms where the the amplitude tensor $A^{\mu\nu}$ has two forms $A^{\mu\nu}_{+}$ and $A^{\mu\nu}_{\times}$ which are orthogonal to each other \cite{Dirkes_2018}. They can be represented as 

\begin{equation}
    A^{\mu\nu}_{+} = h_{+}\, \varepsilon^{\mu\nu}_{+}
\end{equation}

\begin{equation}
    A^{\mu\nu}_{\times} = h_{\times} \,\varepsilon^{\mu\nu}_{\times}
\end{equation}

\noindent where $\varepsilon^{\mu\nu}_{+}$ and $\varepsilon^{\mu\nu}_{\times}$ are unit polarization tensors.

\begin{equation}
\varepsilon^{\mu\nu}_{+} =
\begin{bmatrix}
0 & 0 & 0 & 0 \\
0 & +1 & 0 & 0 \\
0 & 0 & -1 & 0 \\
0 & 0 & 0 & 0 \\
\end{bmatrix}
\end{equation}
\\
\begin{equation}
\varepsilon^{\mu\nu}_{\times} =
\begin{bmatrix}
0 & 0 & 0 & 0 \\
0 & 0 & +1 & 0 \\
0 & +1 & 0 & 0 \\
0 & 0 & 0 & 0 \\
\end{bmatrix}
\end{equation}

\noindent In general relativity any tensor with indices $\mu\nu$ is a rank 2 tensor with 4 rows and 4 columns where each index can take values of space time coordinates which are $(t,x,y,z)$ , and position of each element is associated with any two coordinates. Thus in such tensors, the positions of elements are associated with space-time as follows:

\begin{equation*}
    \begin{bmatrix}
    tt & tx & ty & tz \\
    xt & xx & xy & xz \\
    yt & yx & yy & yz \\
    zt & zx & zy & zz \\
    \end{bmatrix}
\end{equation*}

\noindent So when we compare the unit polarization tensors $\varepsilon^{\mu\nu}_{+}$ and $\varepsilon^{\mu\nu}_{\times}$ with the above one, we see that in $\varepsilon^{\mu\nu}_{+}$ the non zero entries are +1 in $`xx$' direction and -1 in $`yy$' direction, hence the $A^{\mu\nu}_{+}$ amplitude is oriented only along X and Y axes, thus this gravitational wave which oscillates along X and Y axes is called as `PLUS' polarized wave because the vibration resembles `+' symbol. But in $\varepsilon^{\mu\nu}_{\times}$ the non zero entries are +1 in $`xy$' direction and -1 in $`yx$' direction, hence the $A^{\mu\nu}_{+}$ amplitude is oriented in the `XY' plane at a an angle of 45$\degree$ to the axes, thus this gravitational wave which oscillates in the `XY' plane at a an angle of 45$\degree$ to the axes is called as `CROSS' polarized wave because the vibration resembles `$\times$' symbol. 
\\

\noindent So the equation of polarized gravitational waves are:-\\
\begin{centre}
(+) wave $\Rightarrow $  $\tilde{h}_{\mu\nu} = h_{+}\, \varepsilon^{\mu\nu}_{+}\, e^{i(\omega t - k_{z}z)}$ \; and \; $(\times)$ wave $\Rightarrow $  $\tilde{h}_{\mu\nu} = h_{\times}\, \varepsilon^{\mu\nu}_{\times}\, e^{i(\omega t - k_{z}z)}$
\end{centre}
\\
\noindent To simplify things, here position variable is just `$z$', i.e we assume the wave is travelling in z direction and the polarized characteristics are seen in in the X-Y plane. Hence it is easier to figure out the effects of these polarized gravitational waves.

\begin{figure}[h]
    \centering
    \includegraphics[height=3.5cm, width = 7cm]{images.tex/polarization_simulation.jpeg}
    \caption{Simulation of Polarized waves\\ Source:- \url{https://images.slideplayer.com/25/7771045/slides/slide_10.jpg}}
\end{figure}

\pagebreak
 
\subsection{Effect of Gravitational waves on objects}

Gravitational waves carry the fluctuations of space along with them. So if they move through an object, since space itself will oscillate, even the object which occupies space will oscillate according to the wave. Thus the shape of object will change periodically.  

\subsubsection{Plus polarized effect}
When a plus polarized wave passes through the object, since such gravitational wave makes space-time oscillate in X and Y axes only. So the points in space along axis will come close during compression and go far during stretching. Thus the object itself will be compressed and stretched along the axes, perpendicular to the direction of propagation of wave.

\subsubsection{Cross polarized effect}
When a cross polarized wave passes through the object, since such gravitational wave makes space-time oscillate along the line which makes an inclination of 45$\degree$ with X and Y axes (i.e. along the line $x=y$ and $x=-y$). So the points in space along those line will come close during compression and go far during stretching. Thus the object itself will be compressed and stretched along those lines, perpendicular to the direction of propagation of wave.
\\

\begin{figure}[h]
    \centering
    \includegraphics[scale=0.42]{images.tex/effect_of_gw.jpeg}
    \caption{Shape of the object when gravitational wave passes through it when the phase difference of wave changes by $\pi/2$.\\
    Source :- \href{https://inspirehep.net/literature/1787298}{Wavelet graphs for the detection of gravitational waves : application to eccentric binary black holes by Philippe Bacon, Pg 19}}
\end{figure}

\begin{figure}[h]
    \centering
    \includegraphics[scale=0.4]{images.tex/polarization.jpeg}
    \caption{Plus and Cross Polarization of GW. Source :- \href{https://www.researchgate.net/figure/fig4_228909324}{Researchgate.net}}
\end{figure}

\pagebreak

\subsection{Energy transported by gravitational wave}

When sources produce gravitational waves, their energy is converted to Gravitational waves. Since generally the sources are very massive Energy of gravitational waves will be very large. Since gravitational wave travels in the speed of light, the energy is also transported at that speed. And since Energy flux is equal to the product of Energy and speed, The average energy flux `$E$' is given by

\begin{equation}
    E = \frac{c^{3}}{16\pi G} \left \langle (h_{+})^{2} + (h_{\times})^{2} \right \rangle 
\end{equation}

So we see the energy flux is very huge because of the term $\frac{c^{3}}{16 \pi G}$ which is in the order of $10^{33}$ Joules sec/ metre$^2$  and it also depends on the average of the square of the plus and cross polarized amplitudes `$h_{+}$' and `$h_{\times}$'. \\

Due to such huge energy it carries the wave can travel unimpeded forever through space and no obstacle can damp gravitational wave because the space in which the obstacle lies itself is the medium of the wave. But the Doppler effect and decrease in amplitude due to radiation of energy causes the wave to die out after the wave travels a very long distance according to the relation $Amplitude \propto \frac{1}{r}$\:. 
\\

\noindent So the power or intensity of gravitational wave decreases as it moves through space according to this inverse square law i.e. as the wave moves in space through a distance `$r$' The energy of wave will be spread-out in space across a sphere of radius`$r$' through Surface area of sphere $4\pi r^2$. Since the intensity of wave is Energy over time, Intensity reduces as $r^2$

\begin{equation*}
    E_{flux}= \frac{Energy}{Area} = \frac{E}{4\pi r^2}
\end{equation*}

\begin{equation*}
    E_{flux} \propto \frac{1}{r^2}
\end{equation*}

\noindent But since $E_{flux} \propto Amplitude^2$ we get the relation that $Amplitude \propto \frac{1}{r}$. i.e Amplitude decreases as distance from source increases. 

\begin{figure}[h]
    \centering
    \includegraphics[scale=1]{images.tex/inverse_square.jpeg}
    \caption{ Here we see how the intensity of wave changes as it goes farther from the source according to the inverse square law.\\
    \textbf{Source :-} \url{http://www.mysearch.org.uk/website1/html/339.Laws.html}}
\end{figure}

\pagebreak

\section{Sources of Gravitational waves}
Theory of general relativity predicts that gravitational waves can be generated by any dynamically changing system containing moving objects by producing radiation-reaction forces in their source i.e. waves will be generated and carries the exact rate of energy which is extracted from the source. 

Gravitational waves can be produced by an object which is accelerating or a Binary revolving system, merging black holes, neutron star collisions, primordial black holes, etc. But one common nature is all these objects is, they change the curvature of space-time. Thus radiate waves. 
 
\subsection{Single accelerating object in space}
 Accelerating objects like pulsars can create gravitational wave. According to general relativity, mass creates stress in space-time and thus can change the geometry of it by bending and changing the curvature. Then if this object moves, then the curvature also moves along with it. But if the object accelerates in space-time in a circular manner, then the ripples will be created in space-time which is which radiates gravitational waves. This is similar to creation of water waves when we move our finger in a circular fashion in water. So higher the mass of object and it's acceleration, stronger is the gravitational wave it produces.
 
 Such continuously spinning bodies produces continuous gravitational waves, where it's nature is sinusoidal for a longer period of time. This happens only if spin rate of this object is constant. Such gravitational waves have same frequency and amplitude. Such types of gravitational waves are not yet discovered.\\


\begin{figure}[h]
    \centering
    \includegraphics[scale = 0.35]{images.tex/continuous_gw.jpg}
    \caption{Computer simulation of Continuous gravitational wave by a pulsar. \\
    \textbf{Source :-} \url{https://earthsky.org/space/}}
\end{figure}

\begin{figure}[h]
    \centering
    \includegraphics[scale = 0.38]{images.tex/pulsar.jpg}
    \caption{A rotating pulsar.   \textbf{Source}:- \url{https://astronomy.com/news/2018/02/}}
\end{figure}

\pagebreak

\subsection{Revolving Binary Systems}

Revolving binary systems are a very high energy systems which are formed when two massive objects orbit around their common centre of mass called barycenter. Average total mass of such systems are usually greater than 30 solar masses. And average loss of energy per second by such systems will be every huge which will be converted to gravitational waves. Such systems create gravitational waves by a mechanism called `Inspiral' \cite{van_der_Sluys_2008}. There are four phases in this mechanism. \\

\textbf{Interlocking phase} :- This the longest phase where the bodies come closer and get interlocked by their gravity and start to revolve each other around the barycentre. \\

\textbf{Spiral phase} :- Here the objects start getting close as they revolve. Due to the decrease in the distance between them the orbital energy is decreased and this energy is radiated as gravitational wave. But as they come closer and closer, they loose more and more energy, thus the intensity of gravitational wave increases.\\

\textbf{Merger phase} :- During this phase, the bodies collide by producing immense gravitational waves and merge \cite{article}. \\

\textbf{Ring-down phase} :- Finally, the merged bodies become stable and the gravitational wave intensity decreases exponentially and they stop producing gravitational waves. \\

Revolving binary systems produce compact binary inspiral gravitational waves. This is because the intensity of gravitational waves increases slowly during interlocking phase, and exponentially during the spiral phase, then it reaches a peak in merger phase, and finally it decreases rapidly to zero during ring-down phase. The detectors are capable of recording the signal only for a small range of frequency, So in this wave form, the frequency comes to the detecting range and rapidly goes out of range. Thus the signal strength suddenly increases and stops.

\begin{figure}[h]
    \centering
    \includegraphics[scale=0.45]{images.tex/binaries.jpeg}
    \caption{Two massive bodies in inspiral mechanism creating gravitational waves \\
    \textbf{Source :-} https://www.scientificamerican.com/article/gravitational-waves-discovered-from-colliding-black-holes1/}
\end{figure}

\pagebreak

\subsubsection{Binary Black Holes (BBH)}

Black holes are massive objects that can warp space-time extensively. If two black holes get closer and start the inspiral mechanism ,they create ripples in space-time and radiate gravitational waves. Such gravitational waves were the first ones to be detected by LIGO in 2015, September 14th. It was estimated that the collision occurred 1.3 billion years ago, thus the merger occurred 1.3 billion light-years away. This merger was named as `\textbf{GW150914}' meaning Gravitational Wave on 15/09/14. This signal lasted for about half a second.

\subsubsection{Binary Neutron Stars (BNS)}

Neutron stars are dense stars formed by the remnants when a massive star explodes as Supernova. So when two neutron stars merge through inspiral mechanism, they can radiate gravitational waves. First BNS merger was detected on 17th August 2017 and this was named as `\textbf{GW170817}', where the merger was analyzed both by Electromagnetic waves (Gamma ray) and gravitational waves. The signal lasted for comparatively longer duration for about 100 seconds, thus the mass was estimated to be lesser than black holes and was recognised as neutron star merge.



\begin{figure}[h]
    \centering
    \includegraphics[height = 7cm, width = 8.2 cm]{images.tex/GW150914.png}
    \caption{Characteristics of GW150914   Source:-\url{https://www.ligo.org/science/}}
\end{figure}

\begin{figure}[h]
     \centering
     \includegraphics[scale = 0.31]{images.tex/GW170817.png}
     \caption{Characteristics of GW170817\\ Source:- \url{https://www.researchgate.net/publication/233846764}}
\end{figure}

\pagebreak

\subsection{ Gravitational Waves from Collision of Neutron Stars }\\

\subsubsection{The First Detection: GW170817}
 
The stage was set, LIGO's second observation run {02} was coming to a close. When a week was left, on 17th august 2017, the LIGO and Virgo observatory came online and detected a Gravitational Wave (GW170817). Two seconds after the LIGO and Virgo GW detection, the fermi satellite detected a burst of gamma ray radiation (GRB170817A). So loud, you can see it by eye (SNR=32.4). This alerted telescopes and satellites all over the world, which gave the beginning to the most studied astronomical event (-1225 days).  
 
 \subsubsection{The origin of the GW}
 
 $135$ Million years ago, two neutron stars were spiralling faster and faster, stretching and squeezing space-time which created distortions in the fabric, which is gravitational waves. As soon as they reached the distance between Atlanta and Nashfield, they started merging. When they merged it produced a huge explosion of gamma ray radiation. In the last $\frac{1}{10}^{th}$ of a second, the energy released by these stars were $50$x greater than anything else in the universe. Theses waves reached earth after travelling billions of light years in the speed of light. We interpret the component masses of these stars to be between $0.86M\textsubscript{\(\odot\)}$ and  $2.26\textsubscript{\(\odot\)}$, in agreement with masses of neutron stars. The closest and most precisely calculated gravitational wave signal yet. This gravitational wave signalises loudest yet observed, with a combined signal to noise ratio of $32.4$. After the study conducted, it led to look at an area of 28 $\text{deg}^{2}$.\\
 Which led to the discovery of its home NGC 4993. Normally we believe that if neutron stars merged and formed heavier neutron stars which spins rapidly and generates a very strong magnetic field. But it lead to the formation of lowest mass black hole ever found. 
 
\subsubsection{The dynamics of neutron-star binaries and collision}

The initial coordinate separation between the maxima in the rest-mass density is 45 km. The neutron stars inspiral with increasing angular velocity, which deforms each of them tidally. This increases the inspiral rate as it depends on total angular momentum of the system. 

\begin{figure}[h]
    \centering
    \subfloat[]{{\includegraphics[scale=0.2]{images.tex/INSPIRAL.jpeg}}}
    \qquad
    \subfloat[]{{\includegraphics[scale=0.2]{images.tex/MERGER.jpeg}}}
    \caption{(a) Neutron stars in inspiral mechanism. (b) Neutron stars in merger phase.\\ Source:- \href{https://youtu.be/y8VDwGi0r0E}{Neutron Star Merger in YouTUbe}}
\end{figure}

\noindent
During the merger, when regions of the neutron stars with density less than their maximum density come into contact, the tangential components of the velocity exhibit a discontinuity. This can develop an instability called the Kelvin-Helmholtz instability, which leads to the overall amplification of the magnetic field. Such high magnetic fields are seen in magnetars and short hard gamma-ray bursts, which are the consequences of the neutron star collision.

After the merger, the cores of the two neutron stars combine into one and the central rest-mass density starts increasing. The maximum rest-mass density then increases exponentially, and the object collapses to a rotating black hole. This was seen in the detection of gravitational wave- GW170817.

\begin{figure}[h]
    \centering
    \subfloat[]{{\includegraphics[scale=0.23]{images.tex/MERGER2.jpeg}}}
    \qquad
    \subfloat[]{{\includegraphics[scale=0.195]{images.tex/RINGDOWN.jpeg}}}
    \caption{(a) Neutron stars after merging. (b) Neutron stars in ring-down phase. \\ Source :- \href{https://youtu.be/y8VDwGi0r0E}{Neutron Star Merger in Youtube}} 
\end{figure}

In accordance with the dynamics of the neutron-star binaries, during the inspiral, gravitational waveforms increase in amplitude and in frequency. 
The waveforms after the merger have more variation and, in many cases, mostly terminate during the ringdown. The ringdown signal for black holes formed in binary neutron star mergers is at frequencies of the order of kHz and cannot be detected by the present detectors so easily. The post-merger signal is at lower frequencies than the ringdown.

\begin{figure}[h]
    \centering
    \includegraphics[scale=0.74]{images.tex/WAVEFORM.jpeg}
    \caption{Gravitational Waveform after Neutron star collision. Source :- \href{https://www.researchgate.net/figure/Binary-neutron-stars-GW-modes-m-3-2-4-4-6-6-8-8-of-polarization_fig13_233846764}{Research Gate}}
\end{figure}

\pagebreak
 





























































 
\subsection{Primordial Black Holes}
Primordial Black Holes (PBH) are hypothetical black holes thought to have formed in the early universe due to the gravitational collapse of highly dense regions. Since these black holes don't have the usual star as a progenitor, their masses can be lower than the actual mass required for forming a normal black hole. They were first theorized by Yakov Barisovich Zel'dovich and Igor Dmitriyevich Novikov in 1966, and theories of their origin were studied by Stephen Hawking in 1971. \cite{PBH_defn}

\begin{wrapfigure}{r}{5.5cm}
\includegraphics[width=5.5cm]{images.tex/pbh.jpg}
\caption{Computerized image of a primordial black hole.\\
    Source:-\url{https://astronomy.com/news/2019/07/primordial-black-holes}}
\end{wrapfigure} 

\subsubsection{PBH and Stochastic Gravitational Waves}
In the early universe, quantum fluctuations made the inflaton field highly unstable and non-uniform. In rare cases, these fluctuations might have spiked high enough to form energetic peaks which then collapse to form PBHs. This phenomenon would also result in the generation of a stochastic GW (discussed in the next section) background. But, to form such a background, a sufficient number of PBHs are required, which is only possible if the amplitude of the fluctuations are high enough at small scales. \cite{Nakama_2017}\\

Related with PBH and stochastic GW is the concept of cosmic horizon reentry. As the universe expands, represented by the scale fact $a$, comoving length scales\footnote{Comoving length scales/distances are the measure of distances between fundamental observers, i.e, observers that are moving with the expansion of the universe (Hubble Flow), and doesn't change with time.}  between two objects grow along with it. During inflation, $a$ grows exponentially, $a(t) \sim e^{Ht}$, where H is the Hubble constant. But, the horizon stays nearly constant during inflation. Now, the quantity $aH = \dot{a}$ tells us, through its variations in time, whether the comoving length scales grow at a greater or lesser rate than the horizon. During inflation, $a\dot{H} = \ddot{a} > 0$, so the comoving scales grow larger than the Hubble horizon, and after inflation, $a\dot{H} = \ddot{a} < 0$, so the horizon overtakes the comoving lengths, thus an object which was moved outside the horizon during inflation is back inside the horizon. This phenomenon is called Cosmic Horizon Reentry.\\

The survival of oscillation modes of the inflaton field depend on this phenomenon. The modes which leave the horizon and reenter it undergo a 'classical-to-quantum' transition, which transform them into curvature perturbations. These are the perturbations which have the chance to get highly energetic and yield PBH and the stochastic GW. The modes which never leave the horizon don't undergo the transition, so they don't have a major impact on the inflaton field. \cite{CHE}

\subsubsection{PBH and Gravitational Wave Bursts}
If in a small region, numerous curvature perturbations collapse, then a cluster of PBH could form. They dynamics of such PBH is completely different from that of binary PBH systems. Instead of ending up in traditional bound systems and spiral in, majority of PBHs in a cluster would produce a single scattering event via a hyperbolic encounter, only if their relative velocity or relative distance is high enough to escape getting captured into bound systems. Such events would produce bursts of GWs, which can be detected up to several Gpc. \\

In hyperbolic encounters, majority of the energy is released near the closest approach. This has a characteristic peak frequency, which mainly depends on the impact parameter $b$, the eccentricity $e$ and the total mass of the system $M$. These encounters has a duration of the order of a few milliseconds to several hours. GWs from such encounters have very different properties and signature when compared to those from traditional binaries, so detecting them would strengthen the possibility of the existence of PBH. \cite{Garc_a_Bellido_2017}
\pagebreak


\section{Types of Gravitational Waves}
\hspace{0.6 cm} Gravitational waves are produced by each and every body which is accelerating, for e.g. moving cars, air planes, humans, etc. However, they are too small to be noticed and detected with our current technology. In order to study gravitational waves, we need to look at object which are massive and much more bigger than our own solar system like black holes, neutron stars, or huge stars at the end of their lives like gamma ray bursts, pulsars, orbiting black holes, rapidly spinning neutron stars, etc. In fact our universe is filled with many such objects which produce gravitational waves with a significant amplitude and energy. This section was referred by \cite{Allen1996TheSG}, \cite{Neutron_wiki}, \cite{Linear}, \cite{sounds_of_space} \\

\noindent Gravitational wave sources have been divided into following categories:-

\begin{enumerate}
    \item \textbf{Short duration sources :-} Compact binary coalescence, supernovae, gamma ray burst.

    \item \textbf{Continuous sources :-} Pulsars, magnetars, rapidly spinning neutron stars, low mass X Ray binaries, super massive black hole binaries.

    \item \textbf{Stochastic sources :-} Metric fluctuations generated in the very early universe.
\end{enumerate}

\noindent To understand gravitational waves better, LIGO scientists have divided the waves in four categories. The division of the gravitational waves is based on their sources and their characteristic vibration as detected by the interferometers.  They are divided into Continuous, Compact Binary Inspiral, Stochastic and Burst Gravitational waves.

\begin{enumerate}

\item \textbf{Continuous Gravitational Waves} 
    
\noindent Such Gravitational Waves are produced by objects that have constant frequencies and amplitude like a single spinning neutron star. Neutron stars are basically a result of the supernova explosion of a massive star, combined with gravitational collapse, that compresses the core so much that the star density becomes as that of atomic nuclei. Radius of neutron stars will be in the order of 10 kilometers, and mass will be around 1.4 Solar masses. So density will be around $10^{17}\: \text{Kg}\text{m}^{-3}$. The properties of the gravitational waves depend on the spin rate of the star. Therefore, if the spin rate is constant, the properties of gravitational waves (frequency and amplitude) will also remain constant i.e. \textit{continuous}. Spinning neutron stars that possess asymmetric deformations or imperfections in their space produce gravitational waves as it swiftly rotates about its axis. The reasons or effects which can produce asymmetry could be Accretion (large mountain), Magnetic deformations or Pulsar glitches.

\vspace{0.2cm}

\item \textbf{Compact Binary Inspiral Gravitational Waves}
    
\noindent Most of the waves detected so far by LIGO are part of this category. Compact Binary In spiral Gravitational Waves are formed by orbiting pairs of massive objects like neutron stars or black holes. There are three kinds of systems in this category of gravitational wave generators where each one has different characteristics. They are Binary Black Hole system (BBH), Binary Neutron Star system (BNS) and Black Hole-Neutron Star binary system (BHNS). The main phases involved in such systems are spiral, merger and ring-down.

\vspace{1mm}
Let us start by considering two black holes orbiting each other. Initially they are widely separated, spiral occurs over millennia, with each revolution, they emit very weak gravitational waves. Slowly as the energy is lost from the system in the form of gravitational waves, the binary is thus pushed into an orbit with smaller radius and higher orbital frequency. Thus the distance between them decreases and their speeds increase. This causes the frequency of the gravitational waves to increase. Now comes the Merger stage where the black holes come very close and are about to collide to form a single black hole and the black hole thus formed has large distortions in its shape. The strongest gravitational waves are emitted during this process. The distortions thus formed are radiated away as more gravitational waves during the ring-down phase and an undistorted but rotating black hole is left behind. Due to the increase in frequency, the pitch also increases and as a result these gravitational waves would produce a chirp sound. Current results indicate that compact binary objects may well be the most promising sources of gravitational waves rather than supernova collapse.

\item \textbf{Stochastic Gravitational Waves}

These types of waves are generated from random sources, typically arising from a large number of unresolved and uncorrelated events and thus they are the most difficult gravitational waves to detect. Stochastic Gravitational Waves are believed to be the result of processes that took place shortly after the big bang. Just like the Cosmic Micro-wave Background (CMB), these gravitational waves arise from a large number of independent, random events merging to create a cosmic gravitational wave background. Due to their random motion, these waves are the most smallest, that's why the final signal has stochastic nature and the difficult to detect with our current technology. These waves may be analyzed statistically but they cannot be predicted precisely. Detecting these gravitational waves from the Big Bang could allow us to see back in the history of the Universe.

\vspace{0.2cm}

\item \textbf{Burst Gravitational Waves}   

Of all the types of gravitational waves, these waves come from the sources which are yet to be known and thus the form of waves which will be produced is also unexpected. Burst gravitational waves come from short-duration unknown sources. Since we are unfamiliar with these sources, thus its modelling is a big challenge since these will not have well defined properties which are known earlier to us like those of compact binary inspiral waves. Some believe that these waves are produced from systems like supernovae. They are believed to have a 'pop' and 'crack' sound. However, it is difficult to say anything as of now due to the lack of knowledge about their origin. But if we discover an efficient way to detect such GW, revolutionary information about the universe could be revealed.

\end{enumerate}

\begin{figure}[h]
    \centering
    \includegraphics[scale=0.6]{images.tex/Stochastic_gw.jpg}
    \caption{Gravitational Wave Spectrum. Source :- \href{https://link.springer.com/article/10.1007/s41114-017-0004-1}{Detection methods for stochastic gravitational-wave backgrounds by Joseph D. Romano}}
\end{figure}

\begin{figure}[h]
    \centering
    \includegraphics[height = 3.5 cm, width = 10cm]{images.tex/Stochastic wave form.jpg}
    \caption{Stochastic gravitational wave form.\, Source :- \href{https://journals.aps.org/prd/abstract/10.1103/PhysRevD.91.022003}{Searching for stochastic gravitational waves using data from the two co-located LIGO Hanford detectors by J. Aasi}}
\end{figure}

\pagebreak



















































\pagebreak

\section{Why study Gravitational Waves}
\hspace{1cm}Gravitational waves are already used as an important member of multi-messenger astronomy and these can be used to study in depth many objects or phenomena such as:

\begin{itemize}

\item Cataclysmic variables
\item Binary Neutron Stars
\item Young Neutron Stars (the r-mode instability)
\item Low-mass X-ray binaries
\item CMB and Galaxy formation

\end{itemize}

\hspace{1cm}Gravitational waves are emitted by the masses and sent as ripples across spacetime which is completely different from the mechanism of production and transmission of EM waves. Therefore, it could give us more information on the subject matter at hand.

\hspace{1cm}Gravitational waves provide further information about black holes that would otherwise be invisible. Gravitational waves also weakly interact with matter (apart from lensing), thereby reducing energy lost or scattered before reaching the detector. This implies better understanding of inconspicuous regions of space, like the interior of a supernova or the Big Bang.

\subsection{Uses of their Detection}

\hspace{1cm} Gravitational Waves are also used in Astronomy because it allows us to observe the universe the universe in a different way, providing us information about matter such as:

\subsubsection*{Information about the Big Bang}
\hspace{1cm}Gravitational waves have travelled almost unimpeded through the universe since they were generated (which happened $10^{-24}$s after the Big Bang, far earlier than the CMB radiation). Possibilities of non-inflation mechanisms that produce gravitational waves are high. One such possibility could be cosmic strings, which ought to be detectable using gravitational waves. LIGO/VIRGO observations of compact binary in spiral have the potential to bring us far more information than just binary masses and spins:

\subsubsection*{Test the theory of General Relativity}
\hspace{1cm}They can be used for high-precision tests of general relativity. Radiation reaction to some scalar waves in scalar-tensor theories has a signature that can be found with high precision in LIGO/VIRGO.

\subsubsection*{Detection of the Hubble Constant}
\hspace{1cm}They can be used to measure the Universe’s Hubble constant, deceleration parameter, and cosmological constant.Gravitational waves bring a new window to validate the general theory of relativity and cosmological constant as the correct explanation of the theory of gravity and cosmic acceleration. Hubble's Constant shows the expansion of the Universe by showing how distant galaxies are moving away from us and is given by:
 
\begin{equation}
v = H_o*d
\end{equation}

where, v is the Velocity of a galaxy, in $kms^{-1}$

      $H_{o}$ is the Hubble Constant, measured in $kms^{-1}Mpc^{-1}$
      
      d is the Distance of a galaxy, in Mpc

\vspace{1cm}

\begin{figure}[h]
    \centering
    \includegraphics[scale=0.9]{images.tex/HCGW.jpg}
    \caption{Hubble's Constant using Gravitational waves.\\
    Source :- \href{https://www.researchgate.net/profile/Michael-Ross-9/publication/324600496}{A gravitational-wave standard siren measurement of the Hubble constant, Pg 3}}
\end{figure}

\subsubsection*{Polarization of Gravitational Waves}
\hspace{1cm}Gravitational waves carry 2 independent polarizations. A wave will usually have a combination of both. Some sources (rotating) will emit both polarizations with some lag between them. Studying this would give the nature of the source and its rotation.


\subsubsection*{Centrifuge of Binary Stars}
\hspace{1cm}A more careful calculation shows that, for unequal masses, the quadrupole amplitude and the rate of shrinking depend on the masses only through the combination:
\begin{equation}
\ M=\mu^{3/5}*M^{2/5}\
\end{equation}
\hspace{1cm}This is called the chirp mass, where, $\mu$ is the reduced mass and M the total mass. If one can observe, in gravitational radiation, the shrinking time, then one can infer the chirp mass. If one then measures the amplitude of the radiation, the only unknown is the distance r to the source. Gravitational wave observations of orbits that shrink because of gravitational energy losses can therefore directly determine the distance to the source. This is another way in which gravitational wave observations are
complementary to electromagnetic ones, providing information that is hard to obtain
electromagnetically.

\subsubsection*{Spiralling of Black Holes and Neutron Stars}
\hspace{1cm}For a Neutron star or black hole spiralling inwards, the inward spiral has a sort of “map” of it that are emitted in the form of gravitational waves. Analysis of these waves could give information of the body, help determine the type of body(black hole or any other exotic object like a naked singularity)  and can be studied better by LISA for low masses and high frequencies as opposed to LIGO/VIRGO that can capture large masses and low frequencies.


\subsubsection*{Inference from uses of GW}

\hspace{1cm}In summary, we can say gravitational waves provide a new tool for astronomy as well as cosmology and will be ever evolving in terms of types of observations possible ranging from new ways to test for dark matter and the validity general relativity with high precision and observations that simply would not be possible with just electromagnetic waves.This also provides an alternate method to validate the cosmological constant, Hubble’s constant, and various other uses. This section was referred from these 2 papers \cite{Schutz_1999},\cite{Mukherjee_2020}

\pagebreak


\section{Indirect Evidences of Gravitational Waves}
\hspace{0.6cm}Neutron stars are highly compacted core of a dead star, left behind as a remnant of supernova explosion. The pulsars are a unique type of neutron stars that emits beams of electromagnetic radiation out of its magnetic poles. The radiation can be detected from the Earth as blinking star through radio telescopes. The radiation is emitted in the periodic pattern so it appears as a pulsed emission of radiation. The discovery of pulsars was done by Jocelyin Bell a graduate student at Cambridge University in England in the year 1967 \cite{AstronomyAndbeyond:1999} who was working under Antony Hewish. She found out a peculiar pattern in the data in the form of regular pulses. This data was different from the radio signals of the celestial bodies that they detected earlier. At first, they thought that the signal is from some alien civilization so they named them as Little Green Men (LGM) but after few weeks they observed that there were three more objects in the other parts of the sky pulsing with different periods hence they dropped the name Little Green Men and renamed them as Pulsar.\\

Pulsars are among the strangest objects within the universe. Astronomers and scientists use pulsars as an instrument to detect gravitational waves. Pulsars are still found by using large radio telescopes. The largest radio telescope in the world is located at Arecibo in Puerto Rico. A telescope scans the entire sky and scientists look for objects that appear in and out. When a pulsar rotates, it produces detectable pattern of radio emission which is very precise and repeats periodically. Its maximum intensity rises and falls every 23 hours 56minutes. Pulsars spin because the stars from which they are formed also rotate. The slowest pulsar ever detected spins on the order of one per second and the fastest pulsars can spin hundreds of times per second From earth, pulsars often look like flickering stars. On and off, on and off and they seem to blink with a regular rhythm. But the light waves from pulsars doesn't actually flicker or pulse. It radiates two steady, narrow beams of light in opposite direction. The light from the beam is steady and pulsars appear to flicker because they also spin for the same time. \cite{AstronomyAndbeyond:1999}\\

Pulsars are considered to be a great tool in determining the existence of gravitational waves.In 1974 \cite{Linear} the first evidence of gravitational waves was deduced through the motion of the double neutron star system PSRB1913+16. In this system one of the star is a pulsar which emits electromagnetic pulses at radio frequencies precisely at regular intervals as it rotates. Russell Hulse and Joseph Taylor discovered this binary pulsars also noticed that the frequency of pulses shortened and the stars were gradually spiraling towards each other with an energy loss which is closely equal to the energy predicted to be radiated by gravitational waves. For this discovery, Hulse and Taylor were awarded the Nobel Prize in Physics 1993 \cite{Lommen_2015}. Further observations of the binary pulsar and other multiple systems also agree with the General theory of Relativity to high precision. This evidence of gravitational waves is considered as the first indirect evidence of gravitational waves. 
 
\begin{figure}[h]
    \centering
    \includegraphics[scale=0.255]{images.tex/PSR-B191316.jpg}
    \caption{Binary pulsar PSR B1913+16. Source :- \href{https://www.astroblogs.nl/wp-content/uploads/2014/03/PSR-B191316.jpg}{Astroblogs.nl}}
\end{figure}

\pagebreak

\section{Direct search for Gravitational waves}
\input{8 LIGO.tex/8.0 Ligo_intro}
\input{8 LIGO.tex/8.1 Princple}
\input{8 LIGO.tex/8.2 Construction}
\subsection{Working of LIGO}

\subsubsection{Interference}

The name “Interferometer” comes from the principle of working itself i.e. interference.  Interference simply means interfere. When two waves, whether it is a light wave or a water wave, interfere with each other they add up and give some result. That is known as interference pattern. When the crest(trough) of the first wave interfere with crest i.e. peak(or trough i.e. dip) of the other, they interfere constructively. But when the crest falls on the trough of the other wave, it is known as destructive interference. But these are not the only cases. The crest may not fall only on the crest or trough of the other. When watching it graphically, it may be touching somewhere at the middle of the other wave. The simple way of counting it is adding up the amplitude of both the waves. 
\[ A_{1} + A_{2} = A_{12} \]
The resultant amplitude decides the intensity of light that will be produced finally. Intensity of light is directly proportional to the square of the amplitude.
\[ A_{2} \propto I \]
When the resultant amplitude is zero, the intensity is also zero and hence no light is observed. But when the amplitude is other than zero, the intensity is also greater than zero and some light is detected or observed.
\subsubsection{Lasers}
How the laser light is produced and amplified is already well described in the construction part. So, how laser light travels in the whole interferometry will be discussed here. To operate LIGO at its fullest potential, we need an immensely powerful laser. In order to obtain such a laser beam, it must undergo multiple stages of amplification. An Nd: YAG laser after passing through Non-Planar Ring Oscillator (NPRO), Master-Oscillator Power Amplifier (MOPA), which consists of four thin rods of Nd: YVO4, and finally through High-Power Oscillator (HPO), a powerful laser of power 200W, with wavelength 1064nm is obtained. This light beam incidents on a power recycling mirror followed by a beam splitter. The partially reflecting beam splitter splits the light into the 4km long ultra-high vacuum chambers orthogonal to each other. Fabry-Perot cavities are included in these chambers to improve the sensitivity. The light circulates for around 300 times between the test mass mirrors in these Fabry-Perot cavities. This boosts the laser power to 100kW. 
The light in both cavities, after circulating, should recombine at the beam splitter leaving the arms. These light waves will be in coherence; hence they cancel out each other letting no light to reach the photo-detector. However, this is not the scenario if a gravitational wave passes nearby.

\subsubsection{Effect of gravitational waves on the interference pattern}

Generally, gravitational waves stretch the space in one direction and compress it in the normal direction. Hence, when the gravitational wave passes through the LIGO, the orthogonal arms will be subjected to a strain, i.e. their lengths are increased or decreased. If one cavity gets extended, the other cavity gets shortened. This causes a variation in the effective length travelled by the laser beam. The light from shorter arm reaches the beam splitter prior to the light from longer arm creating a small phase difference. Due to this resonance occurs and some light reaches the photo-detector. This gives out the signal indicating a gravitational wave. The light which did not reach the photo-detector returns to beam splitter followed by power recycling mirror to boost the laser.
But apart from gravitational waves, there are other disturbances also. Like earthquakes, trucks going through a road nearer to the interferometry can also cover the influence of GW on laser light which is as minute as 1/10000th the width of a proton. And hence to avoid this as much as they can, the whole system is evacuated to the trillionth part of the outside atmosphere. Other measures are also taken to dampen these vibrations which are described in this paper later.
 
\pagebreak


















































































\pagebreak

\subsection{Noise and it's Cancellation by LIGO}

Any disturbance in the surrounding could potentially act as noise for LIGO which tends to interfere with the detector and could produce it's own signal instead of a gravitational wave.

\subsubsection{Seismic Noise}

A gravitational wave is measured by monitoring the relative distance between two test mass surfaces. Any force affecting the centre of mass would then result in an ambiguous and faulty measurement. A ground-based interferometer is mechanically coupled to the earth and the masses are prone to seismically driven vibrations. The dominant part of the seismic power spectrum is at low frequencies. A moderately quiet site will have a spectrum of roughly,

\begin{equation}
    x(f) = \frac{1}{f^2} \times 10^{-8}m(Hz)^{3/2}
\end{equation}

where $x(f)$ is Band-width of the noise as a function of frequency($f$).

\begin{enumerate}
    \item Passive technique of seismic noise reduction \\ 
    
    It utilizes the inertial response of a mass on a spring. Passive isolation takes advantage of the fact that above the resonant frequency, $f_0$, of the mass-spring system, the response of the mass to driving forces decreases by $(\frac{f_0}{f})^2$ . Systems with lower resonant frequencies give higher isolation at a given frequency. These passive systems can also be staged by suspending one isolation system from the isolated stage of a previous system. The total isolation then is the product of each mass-spring system, $(\frac{f_0}{f})^{2n}$, where `$n$' is the number of stages (assuming the same resonant frequency).
    
    \item Active technique of seismic noise reduction \\
    
    Active isolation techniques employ a bootstrapping method. A proof mass is placed on the platform being isolated. The proof mass is more inertial than the platform it sits on. Monitoring the relative displacement, velocity, or acceleration between the platform and the proof mass generates an error signal when the platform has suffered a disturbance to its state. Feedback control systems are used to correct the error signal, locking the position of the platform to the inertial reference of the proof mass. \\
    The level of isolation is proportional to the closed-loop gain of the system when the sensor noise is low enough. The limits to the closed-loop gain, the isolation are the sensor’s bandwidth and noise. An advantage to arranging the isolation system in stages is that loop gain in each stage can be more modest, which is sometimes forced by the available bandwidths and mechanical resonances of the structure. LIGO I uses a simple, multi-layer passive isolation system which places a “wall” in the seismic noise spectrum at roughly 40 $Hz$. The proposed LIGO II seismic isolation is largely active \cite{giaime2000active}. A quiet hydraulic system is used externally to the vacuum chambers which house the test masses. This external system has large dynamic range, and is used primarily to take-out long-time scale drifts and disturbances. A two-stage active isolation system inside the test mass chambers is supported by the external system through bellows. The active system isolates an optical table in all six 7 degrees of freedom, from which the test mass is hung as the lower mass of a quadruple pendulum. This design is expected to move the seismic wall to $ \sim 10 Hz$
\end{enumerate}

\subsubsection{Thermal Noise}

Another noise source is due to the fact that the masses are at finite non-zero temperature. Non-zero temperature dictates that the atoms which comprise the masses, as well as the wires which suspend the masses, vibrate, according to entropy. Vibrations of the test mass atoms cause the surfaces of the mirrors to vibrate, which generates a signal. Thermal noise affecting the surface of the test mass directly is called internal thermal noise while the thermal noise from the suspension wires is called pendulum thermal noise.

\subsubsection{Optical noise}

The last of the fundamental noise sources is a limitation of the measurement process itself. The measurement process involves the interaction of light with the test masses, and the subsequent counting of the signal photons by a photo-detector. This has traditionally been thought of in terms of two uncorrelated sources - the Poissonian statistics of the counting of photons, otherwise known as shot noise, and the Poissonian statistics of the force on the test masses from photons, known as radiation pressure noise. \cite{weiss1972electronically} , \cite{saulson1994fundamentals}\\
Shot and radiation pressure noise are manifestations of the two quadrature of the vacuum. The square-law photo-diode measures the product of the amplitudes of the vacuum and the coherent light from the laser. Increasing the laser power increases the shot noise sensitivity while the radiation pressure noise sensitivity decreases. In LIGO I, 6 watts of light are incident on the interferometer, and radiation pressure noise is negligible. In LIGO II, however, 120 watts of power are planned, making radiation pressure an important factor. The shot noise spectral density is flat, while the radiation pressure amplitude spectral density has a 1/f shape. At a given frequency, the quadrature sum of the shot and radiation pressure noise can be minimized by using the right amount of power. This defines the standard quantum limit. \cite{braginsky1995quantum}

\begin{equation}
    h_{SQL}(f) = \sqrt{\frac{8\hbar}{(2\pi f)^2 mL^2}} 
\end{equation}
where the minimum level of quantum noise is defined as a function of frequency `$h_{SQL}(f)$', `$m$' is  the mass of the test mass, `$L$' is the length of the interferometer arms, `$\hbar$' is the reduced plank's constant. This is actually a locus of the optimum strain spectral density at frequency `$f$' assuming the optimized input power for that frequency,
\begin{equation}
    P_{SQL} = \frac{mL^2(2\pi f)^4}{4\omega_0}
\end{equation}
where $\omega_0$ is the angular frequency of the light. This limit makes the assumption that the shot noise and the radiation pressure noise are uncorrelated. Recent work has discovered that there are correlations in the radiation pressure noise in signal tuned interferometers. \cite{buonanno18optical},\cite{buonanno2001quantum}

\pagebreak















































































\pagebreak

\subsection{Signal Extraction}
Once the excess noise has been diminished, the gravitational wave signal needs to be identified and pulled. Excess power and Template matching are some of the more frequently used methods to identify and extract the signals.

\subsubsection*{Excess power method}

The “excess power” method is much enhanced when several detectors are employed. With it, the data streams from each observatory are searched for signals that are not easily accounted for by the noise characteristics of that particular instrument. When such interesting signals are found, corresponding signals, using an appropriate time window, are searched for in the data from the other observatories. Essentially, the signals from the different detectors are cross-correlated with each other. Since the noise in each experiment is uncorrelated with that of the others, a real signal should give a large spike in the correlation statistic. Complicating the analysis is the sensitivity of the detectors to the direction to the source, which can weaken the signal in one observatory relative to the others. But this effect can be taken into account and poses no serious problem.

\subsubsection*{Template matching}

Another method to find gravitational waves is to look for signals that coincide with events that are visible using other means, and that should also emit gravitational waves. The first step in computing a template signal is computing the gravitational wave signature of different astrophysical sources. While template matching is a powerful way to extract a gravitational wave signal from the noise, it only works for sources that can be easily modelled. \\

The models compared to the LIGO data are called phenomenological models, and they are fit to numerical simulations of systems created by solving the Einstein equation on supercomputers. A smaller number of simulations are computed, and then an analytic model is created that links one simulation to another. A set of freely adjustable parameters is used with these models that allow them to match all of the available numerical simulations and to interpolate between them. There are several models that are used for this. Each uses a slightly different method, and so they produce slightly different wave forms. The implied properties of the modelled systems also differ as a result, but only in small ways. However, template matching alone would result in a lot of spurious matches and random data fluctuations.

\begin{figure}[h]
    \centering
    \includegraphics[scale=0.92]{images.tex/template_matching.jpg}
    \caption{Matching the received signal using Gravitational waveform as a template.\\ Source :- \href{https://kiss.caltech.edu/workshops/LISA/presentations/Babak.pdf}{Introduction to Data Analysis of Gravitational Wave Signals by Stanislav Babak}}
\end{figure}

Therefore, we need to apply a “SNR” filter13, taking only the samples above some threshold value of the signal-to-noise ratio. In the LIGO data, it turns out that if one sample has a high SNR, then it is usually surrounded by neighbors that also have high SNR. Most of these are false positives and LIGO performs yet another cut, keeping only the single highest SNR candidate in each such cluster. This further reduces the size of the data set. After that, the software looks for coincidence between the two LIGO antennas. From the reduced data set, a statistic called the log likelihood ratio, or LLR is computed . The larger the LLR for a candidate, the more likely it is to be caused by a signal rather than noise, and vice versa. When a real signal is present in the data, it is generally surrounded by a large number of candidates which are the result of matches by similar-shaped templates to the matching one. This has been determined by many runs in which simulated waveform data have been injected into the data stream to test the software. \\

This allows candidates to be clustered, similar to the way the SNR threshold peaks were clustered; in this case, if a candidate falls within 4 seconds of another candidate that has a higher LLR, the weaker candidate is discarded. It turns out that there is a high probability of finding a low-ranked candidate next to a high ranked one, and so this method successfully trims many low ranked candidates from the sample. However, for candidates with LLR larger than about 6, this clustering method is not effective because the probability of two such candidates being within 4 seconds of one another so low: the recent data run produced one candidate like this about every 5 minutes, or 10, 000 of them for the entire run. The total sample in the data set has now been reduced from 500 million per second to only 10, 000 total, a small enough sample that it is possible to perform detailed statistical testing on each one of them.\\

Once the GW signal is extracted, then it's frequency will be changed by multiplying it with a conversion ratio, such that the resulting frequency lies in the audible range. On September 14 2015, the first characteristic `chirp' sound of the GW150914 was decoded using LIGO. 

\pagbreak


\section{Detection of Gravitational waves using LIGO }
\subsection{Discovery of the the First Gravitational Wave}

\hspace{0.5cm} GW150914 was detected for the first time by the two detectors of LIGO at 09:50:45 UTC on the 14th of September, 2015 \cite{PhysRevLett.116.061102}. The signal received opened up a gateway with deeper understanding of astronomy and particle physics \cite{Abbott_2016}. The source was discovered to be of a binary black hole coalescence. This detection serves to be groundbreaking in terms of both GW and binary black hole systems.

The possibility of detecting gravitational waves were feeble due to the technology available during Einstein’s theory of relativity, although, experiments in search for the signal began in the 1960s with resonant mass detectors.  The interferometers were suggested in the 60s and 70s, finally by 2000s they were set up.   
The evidence for the presence of GW was observed by Hulse and Taylor by the discovery of a binary pulsar system PSR B1913+16.1 The system depicted subsequent loss of energy. \cite{PhysRevLett.116.061102}\\

\subsubsection{Source of GW150914}
\hspace{0.5cm} The source of GW150914 is a binary black hole merger. On analysis it was theorized that the two black holes were an undisturbed binary star system whose approximate masses were 36 and 29 solar masses successfully collapsing into a single black hole \cite{Abbott_2016} . Studies suggested that the mass of the system decreased considerably after the merger, indicating the emission of gravitational waves \cite{Ligo_org},\cite{LIGO_org}. From the merger, energy with three times the mass of our sun was converted into gravitational wave energy \cite{LIGO_org}.

This system is located 1.3 billion light years away from our solar system. The coalescence produced tremendous power and energy during the final 20 milliseconds of the merger. The increase in their tangential velocity to 60 percent the speed of light, the short separation of 350km between them, orbital frequency of 75 Hz, half the gravitational frequency of 150 Hz, confirms the signal to be from a merger of two enormous black holes because no other compact objects other than black holes can come that close without merging, not even neutron stars as they wouldn't have the required mass. \cite{PhysRevLett.116.061102},\cite{LIGO_org}\\

\subsubsection{Detection of GW150914}
\hspace{0.5cm} The two LIGO detectors at Washington State and Louisiana received the GW150914 signal, however, they were running in engineering mode. Hence, it required a 16 day analysis to confirm the signal to be legitimate and not a test simulation \cite{LIGO_org}. In order to confirm its validity, the environment detectors were checked to have no disturbances having similar properties as the GW150914 signal. At the time, LIGO was the only observing detector, the Virgo detector was not functional since it was being upgraded while GEO 600 was not sensitive enough to catch the signal \cite{PhysRevLett.116.061102}.

The LIGO detector at Hanford suffered a 7 millisecond delay than Livingston. The signal was processed in only 3 minutes after detection. It lasted for 0.2 seconds during which its frequency increased in 8 cycles from 35 Hz to 150 Hz. By Signal conversion process, when the signal was converted, it was in the audible range and created a noise similar to the chirp of a bird and was termed as the chirp signal \cite{PhysRevLett.116.061102},\cite{LIGO_org}.  

The LIGO detectors had successfully detected the gravitational wave signal emitted from a binary black hole system in 2015. It was a successful prediction of the general theory of relativity. The observations served influential in terms of both the signal as well as existence of binary black hole mergers \cite{PhysRevLett.116.061102}.

\pagebreak
















































































\pagebreak
\subsubsection{GW190814}

Two advanced-LIGO detectors (Hanford, Washington and Livingston, Louisiana, USA) and the advanced-Virgo detector (Cascina, Italy), have detected gravitational waves from the inspiral and merger of a stellar-mass black hole and another compact object on 14th August, 2019 at 21:10:39 UTC. It has been named as GW190814 as the date suggests.

\begin{figure}[h]
    \centering
    \includegraphics[scale = 0.91]{images.tex/GW190814.jpg}
    \caption{Frequency Vs Time data of GW190814 in three observatories. Source:- \href{https://en.wikipedia.org/wiki/GW190814}{Wikipedia}}
\end{figure}

While the mass of one component of this binary could range from 22.2 to 24.3 $M_\odot$ black hole, the other component which was of 2.6 solar mass could be either a low-mass black hole or a heavy neutron star. The masses of the objects before merging differed by a factor of 9. This makes it the most extreme mass ratio known for some GW event. The source of this GW was in a small patch of sky of around 20 square degrees. Even after doing so much research, the counterpart of the black hole which was in the inspiral mechanism wasn’t observed. It can be that, either black hole consumed the neutron star completely or both were black holes. Had we observe an electromagnetic counterpart, which may not have happened due to a number of reasons, we could say the smaller object is mostly neutron star. 

\pagebreak

\subsection{GW170817}

On 17th August, 2017, LIGO and Virgo detectors observed a gravitational wave named as GW170817. It is known to be produced by two neutron stars merging into each other while spiralling closer and closer. The aftermath of this GW was observed by around 70 observatories on 7 continents as well as through space, across the electromagnetic spectrum, marking a significant breakthrough for multi-messenger astronomy. The discovery and subsequent observations of GW170817 got Breakthrough of the Year award for 2017 by the journal Science.

\begin{figure}[h]
    \centering
    \includegraphics[scale=0.78]{images.tex/GW170817_observatories.png}
    \caption{Frequency Vs Time data of GW170817 in three observatories. Source:- \href{https://en.wikipedia.org/wiki/GW170817}{Wikipedia}}
\end{figure}

The component masses of the binary are inferred to be between 1.17 and 1.60 $M_\odot$. After merging it makes the mass of about 2.74 $M_\odot$. The gravitational wave signal lasted for about 100 seconds. It started with a frequency of 24 $Hz$. It  inspiralled for around 3,000 cycles. The amplitude and frequency increased to a few hundred hertz as both the objects came nearer in the typical inspiral chirp pattern. Lastly, it ended with the collision at 12:41:04.4 UTC which was received as a signal in the interferometer. At first, it arrived at the Virgo detector in Italy. After 22 milliseconds, detectors at the LIGO-Livingston detector in Louisiana, United States got the signals. After another 3 milliseconds, the waves reached at the LIGO-Hanford detector in the state of Washington, United States. It was then compared with a prediction from the general theory of relativity given by Einstein to analyse it further. The source was localised within a sky region of 28\degree square which has a probability of 90\%.

A gamma-ray burst, GRB 170817A was detected. It lasted for $\approx$ 2 seconds. It was detected by Fermi and INTEGRAL spacecrafts. This bursts began at 1.7 seconds after the signal received denoting the merge of the objects. It's a hypothesis that neutron star mergers cause gamma-ray bursts which gets confirmed with this merger.
















































































\pagebreak


\section{Advanced LIGO and other Gravitational wave interferometers}
\input{10 Advancement}

\section{Conclusion}
\input{11 Conclusion}

\printbibliography
\end{document}
